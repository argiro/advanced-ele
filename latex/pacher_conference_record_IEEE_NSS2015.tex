%% bare minimum for a IEEE-like abstract
%%\documentclass[a4paper, 9pt, twocolumn]{article}

%%\documentclass[conference]{IEEEtran}   --include-directory=C:\path\to\IEEEtran directory 
\documentclass[conference]{./templates/IEEEtran/IEEEtran}


%% define page margins
%\usepackage{geometry}


\usepackage{upgreek}
\usepackage{bm}

%\usepackage{textcomp}  %% for \mu => \textmu
%%\usepackage{palatino}

%%\usepackage{siunitx}

%% page margins
%\geometry{verbose,a4paper,lmargin=15mm,rmargin=15mm,tmargin=20mm,bmargin=30mm}

%% column separation
%\setlength{\columnsep}{8mm}

%% language
%\usepackage[english]{babel}


%% additional packages
\usepackage[pdftex]{graphicx}
%%\usepackage{nomencl}
%%\usepackage{booktabs}
%%\usepackage{latexsym}
%%\usepackage{pdfpages}
%%\usepackage{color}
%%\usepackage{verbatim}
%%\usepackage{mathrsfs}
%%\usepackage{draftwatermark}


%% nice style (bold numbering)
%\usepackage{fancyhdr}

%\pagestyle{fancy}
%\fancyhf{}
%\fancyhead{}

%%\fancyhead[LO]{\bfseries \nouppercase{\leftmark}}
%%\fancyhead[RE]{\bfseries \nouppercase{\rightmark}}

%% bold page numbers
%\fancyfoot[CO,CE]{\bfseries \thepage}

%\renewcommand{\headrulewidth}{0pt}



% correct bad hyphenation here
\hyphenation{ca-pa-ci-tance compa-risons}

%% begin document
\begin{document}
% Don't want date printed
\date{}

%% paper title
\title{\LARGE {\bf A Low-Power Low-Noise Synchronous Pixel Front-End Chain              \\
                       in~65~nm CMOS Technology with Local Fast ToT Encoding        \\ 
					   and Autozeroing for Extreme Rate and Radiation at HL-LHC} }


%%Luca Pacher\thanks{work}
                  
\author{Luca Pacher$\,^{1,2}$\footnote{Corresponding author}, \emph{Student Member IEEE}, Ennio Monteil\,$^{1,2}$, \\
        Angelo Rivetti$\,^{2}$, \emph{member IEEE}, Natale Demaria$\,^{2}$ and Manuel Da Rocha Rolo$\,^{2}$        \\

\vspace{1.6mm}	\\	 

\small{$^{1}$ University of Torino, Department of Physics, 10125, Torino, Italy}                        \\[1mm]
\small{$^{2}$ Istituto Nazionale di Fisica Nucleare (INFN), Section of Torino, 10125, Torino, Italy }   \\[2mm]
%%\small{$^{\dagger}$ corresponding author - pacher@to.infn.it }	 		

}%% end author

\maketitle
  
%% abstract			
%%\section*{\centering \normalsize Abstract }

\section*{  }
\begin{abstract}
{\bf A low-power and low-noise synchronous front-end chain in a commercial 65 nm CMOS technology
suitable for the future pixel upgrades at the CERN Large Hadron Collider (LHC) is 
presented. A~shaper-less Charge-Sensitive Amplifier (CSA) with constant current feedback
provides triangular pulse shaping for linear Time-over-Threshold (ToT) 
charge measurement. The sensor leakage current is compensated 
by the same feedback network. A~track-and-latch voltage comparator 
is adopted for the hit discrimination. The hit generation is synchronized with a 
40~MHz clock, minimizing time-walk issues in the time-stamp assignment. 
Fast ToT charge encoding up to 8-bit resolution can be retrieved at the pixel 
level exploiting a high-frequency self-generated clock signal. This is obtained by turning 
the latch into a voltage-controlled oscillator (VCO) using asynchronous logic. 
Pixel-to-pixel threshold variations are compensated by means of an autozeroed 
scheme, thus avoiding the need of a on-pixel D/A converter.
An array of 8~$\times$~8 cells with 50~$\bm{\upmu}$m $\times$ 50~$\bm{\upmu}$m pixel size has been prototyped.
Design specifications, implementation and test results are discussed.}
\end{abstract}



%%\vspace*{5mm}

\vspace*{1.2cm}

%% summary
\section{Introduction}


\noindent The foreseen High-Luminosity (HL) LHC upgrade \cite{Rossi2012}
\let\thefootnote\relax\footnote{This work was partially 
supported by the Italian Ministry of Education and Research 
(MIUR) under contract 2012Z23ERZ.}
will impose the installation of new silicon pixel detectors 
in the inner tracking systems of general-purpose experiments. 

With increased performance the machine will deliver proton
collisions with an instantaneous luminosity up to 10$^{35}$~cm$^{-2}$s$^{-1}$, 
one order of magnitude higher with respect to the current design value, targeting 
to reach an integrated luminosity of 3000~fb$^{-1}$ in 10 years.
With such a luminosity and a centre-of-mass energy of 14~TeV the nominal collision 
rate of 40~MHz will lead to unprecedented track densities, introducing extreme rates and radiation levels.
More layers equipped with sensors featuring high granularity, speed and radiation 
hardness will be required close to the interaction regions. Thus hybrid silicon pixel detectors 
will continue to play a fundamental role.

The innermost pixelated layer will have to cope with an expected Total Ionizing Dose (TID)
of 10 MGy in 10 years, corresponding to 2~$\times$~10$^{16}$ (1~MeV) $n_{eq\,}$/cm$^{2}$. 
Smaller pixels of the order of 50~$\mu$m~$\times$~50~$\mu$m will 
be required to maintain high spatial resolution and two-tracks separation. 
The particle flux will increase to 500~MHz/cm$^2$, leading to hit rates of the order of 3~GHz/cm$^2$
and an estimated average rate per pixel of 75~kHz. The foreseen usage of thinner sensors of 
100-150~$\mu$m thickness to increase the radiation tolerance will determine reduced signals, 
needing low-threshold (as low as 1~k$e^-$ minimum detectable charge)
and low-noise performance for the analogue front-end (below 150~$e^-$ RMS at nominal 100~fF input capacitance including strays).
Moreover a time response below 25~ns is required in order to cope with the nominal 
LHC bunch crossing rate, while keeping bias currents to acceptable 
values and targeting to a total maximum power dissipation of 10~$\mu$W per pixel. More on-chip
intelligence and much higher readout bandwidth will be required to accommodate unprecedented data rates. 


%% ASICs
Research and development activities devoted to the design of new pixel Application Specific Integrated 
Circuits (ASIC) suitable for HL-LHC upgrades have started. 
A commercial 65~nm CMOS process has been identified by the pixel ASIC community 
as a promising fabrication technology for the implementation of new generation pixel readout chips. 
Such a 65~nm was already demonstrated to be radiation tolerant up to 3~MGy \cite{Bonacini2011}. 
Technology qualification and radiation hardness studies using 65~nm CMOS are now part of the 
international RD53 collaboration research program officially supported 
by CERN \cite{Christiansen2013} and of the Italian INFN CHIPIX65 project.

Preliminary pixel front-end test structures, small pixel arrays and 
other analogue, digital and mixed-signal building blocks have been submitted 
by the CHIPIX65 collaboration to the foundry access service. 
They were received back from the manufacturer for laboratory test measurements 
and bench characterizations at the beginning of 2015.


%% pixel chain Torino
A small pixel array of 8~$\times$~8 cells with 50~$\mu$m~$\times$~50~$\mu$m 
pixel size has been prototyped as part of the first 
CHIPIX65 submission. In order to gain from increased speed
offered by a 65~nm CMOS technology node and meet the low-threshold and low-power requirements, 
a~synchronous front-end architecture has been implemented. 
Section II of the paper describes the ASIC design, while test results are presented in Section III 
and conclusions are drawn in Section IV.



\vspace*{1.5cm}



\section{ASIC design}
\noindent A schematic block diagram of the front-end chain is reported in Fig.~\ref{archi}.
The input stage is a charge-sensitive amplifier implemented as a single-ended,
60~dB open-loop gain inverting amplifier with two selectable feedback capacitors.
In order to save power and area a shaper-less solution is chosen. Hence the 
CSA output directly drives the front-end discriminator.



\begin{center}
\begin{figure}[!htpb]
\centering
\includegraphics[width=0.48\textwidth]{./pictures/schematics/architecture.png}
\caption{Schematic block diagram of the front-end channel. Selectable shunt capacitances 
         at the input node allows to mimic different values of sensor capacitance.}  
\label{archi}
\end{figure}
\end{center}



 Linear charge measurements 
up to 40 k$e^{-}$ are performed using the ToT technique, therefore triangular pulse shaping is adopted.
A time-invariant feedback network based on an auxiliary transconductance amplifier discharges the feedback capacitance 
with a selectable constant current in the 5-50~nA range after a charge signal has been detected~\cite{Krummenacher1991}.
The~same feedback circuit compensates also the sensor leakage currents up to 50~nA.
A calibration circuit is used to inject a test charge at the CSA input node. Selectable test
capacitors have been added to mimic different values of pixel sensor capacitance.


%% DISC
The front-end amplifier is AC coupled to a discrete-time hit discriminator. In thi way, any offset caused by the feedback 
transconductor can not propagate to the discriminator.
A track-and-latch voltage comparator is employed. It is implemented as
a low-gain high-bandwidth differential preamplifier coupled to a regenerative latch stage. 
Thanks to positive feedback, precise and fast voltage comparison can be obtained with 
low-power dissipation, allowing to discriminate very low charge-induced signals above 
the nominal threshold. The generation of a CMOS digital hit pulse is synchronized with a
40~MHz master clock, sampling the CSA analogue output. This provides
a reliable solution that greatly relaxes time-walk issues in the time-stamp assignment.

The latch stage is shown in Fig.~\ref{latch}. To reduce the power consumption, a dynamic architecture 
is used, thus avoiding any DC path between power and ground rails both in the reset state and after 
regeneration has occurred. Transistors M$_3$ and M$_4$ are inserted to isolate the latch from its driving preamplifier 
(not shown in the figure) before the full output swing is attained. This significantly reduces the propagation 
of kickback noise towards the charge-sensitive amplifier.



\begin{center}
\begin{figure}[!htpb]
\centering
\includegraphics[width=0.48\textwidth]{./pictures/schematics/latch.png}
\caption{Regenerative stage of the synchronous comparator. Transistors M$_3$ and M$_4$ are introduced to mitigate 
         the kickback noise.}
\label{latch}
\end{figure}
\end{center}



%% fast TOT
The latch can be turned into a fast voltage-controlled oscillator (VCO) 
by means of a dedicated asynchronous logic. Similar techniques are employed 
in the design of modern charge redistribution Successive Approximation 
Register (SAR) A/D converters which internally generate the necessary clock signals 
for SAR operations~\cite{Liu2010}. As~derived from transient simulations,
flexible and high-speed ToT digitizations up to 8-bit resolution can
be performed in less than 400~ns using selectable on-pixel 
self-generated clock waveforms in the 100-900 MHz range.

%% autozeroing
Pixel-to-pixel threshold variations are compensated without the need 
of a local D/A converter for digital trimming. The offset voltage is periodically 
sampled and stored on capacitors using Output-Offset Storage (OOS) between the 
preamplifier and the latch~\cite{Razavi1992}. The lack of a on-pixel D/A converter 
introduces fundamental advantages in perspective of pixel operations in a harsh radiation 
environment, avoiding the necessity of dedicated Single Event Upset (SEU) tolerant registers 
to store the configuration bits for digital trimming. The available area for local temporary 
data storage (buffering) and signal processing in the digital part can significantly increase. Furthermore 
efficient calibration schedules can be defined according to online machine operations. 

The final layout of the chip, referred to as CHIPIX/TO, is presented in Fig.~\ref{CHIPIX_VFE1_TO_final}. 
The analogue part occupies about 50\% of the total pixel size.


\vspace*{0.7cm}

\section{Test results}
\noindent A picture of the prototype wire-bondend on the test board. In the measurements, the digital 
control signals are provided by a data pattern generator and the outputs are acquired with a fast, 
large bandwidth digital storage oscilloscope. Both the CSA and the discriminator outputs can be inspected 
independently.





\begin{center}
\begin{figure}[!htpb]
\centering
\includegraphics[scale=0.28]{./pictures/layout/pixel_matrix_1x1_grayscale.png}
\caption{Complete layout of the CHIPIX/TO chip, 945~$\mu$m~$\times$~945~$\mu$m. 
         The~prototype contains 8~$\times$~8 pixels with synchronous front-end and full-analogue 
		   readout of all channels.}  
\label{CHIPIX_VFE1_TO_final}
\end{figure}
\end{center}



\begin{center}
\begin{figure}[!htpb]
\centering
\includegraphics[scale=0.4]{./pictures/setup/wirebonded_gray.jpg}
\caption{Prototype chip wire-bondend on the test PCB.}  
\label{setup}
\end{figure}
\end{center}





% conclusions
%The choice of a synchronous front-end discriminator introduces therefore several advantages and
%very promising features that naturally fits into a bunched experiment.


Sample oscilloscope waveforms are presented in Fig.~\ref{oscilloscope}. In the upper plot, 
the CSA response to a test input of 1.5~fC is shown together with the synchronous hit signal. 
In the lower one, the fast clock generated internally to the pixel is reported. This clock can be 
used to drive a fast counter that encodes the signal duration. The on-chip output CMOS driver limits 
the maximum measurable frequency to about 150~MHz. 

%The measured Equivalent Noise Charge (ENC) is about 100 $e^{-}$ RMS at nominal 100~fF
%input capacitance. The total power consumption is 6.4 $\mu$W per pixel. 
%Measurements are in agreement with CAD simulations. First irradiation tests with 
%X-rays and 3 MeV protons are foreseen in summer 2015.



In the system, a trade-off exists between signal-to-noise ratio and dead time due to the choice of the bias current in 
the feedback transconductor. In fact, a higher current determines a faster return to the baseline, but induces also an 
additional ballistic deficit that affects the SNR. This aspect can be appreciated in Fig.~\ref{noise} that reports the ENC 
versus input capacitance for 10~nA and 40~nA feedback current. For a typical input capacitance of 100~fF, the noise is 
respectively 90 and 140 electrons RMS. Due to the differential nature of the  transconductor, only half of the feedback 
current is actually available to discharge the CSA feedback capacitor. The dead time for an expected average signal of 
2~fC is therefore 125~ns


\begin{center}
\begin{figure}[!htpb]
\centering
\includegraphics[scale=0.31]{./pictures/oscilloscope/hit_10ke.png}
\includegraphics[scale=0.31]{./pictures/oscilloscope/fastClk_10ke.png}
\caption{Sample waveforms at the oscilloscope for 10 k$e^{-}$ injected charge and 40~nA feedback current. 
         The CSA output pulse with triangular shaping is sampled at 40 MHz. 
         The width of the hit pulse is an integer number of clock cycles (top).
         Latch operations when turned into a local oscillator for fast ToT counting (bottom). 
		 The time base is 50 ns/div.}
\label{oscilloscope}
\end{figure}
\end{center}



\noindent for 40~nA and 500~ns for 10~nA.  The first value is already adequate for operating at the maximum 
expected rate of 75~kHz per pixel with an efficiency better than 99\%.



Before trimming the measured discriminator offset is 250 electrons RMS, as presented in Fig.~\ref{untrimmed}.
The trimming is done by shorting both discriminator inputs to a common voltage and storing the offset in coupling 
capacitors that connect the low-gain differential amplifier to the latch. For maximum accuracy, the offset 
compensation cycle should be repeated every 100~$\mu$s. A few issues in the offset-compensation mechanism were 
detected in the first version of the prototype and fixed in a second revision. 

The threshold dispersion after trimming, measured in the improved prototype, is shown in Fig.~\ref{trimmed}. 
The residual offset after calibration is 69.9 electrons RMS and agrees extremely well with the value (70 electrons) 
predicted with computer simulations to evaluate the latch dynamic offset, which represents 
the most prominent contribution. A minimum threshold of 600 electrons can therefore be used once the system is calibrated.
The full front-end dissipates 6.4~$\mu$W at 1.2~V supply voltage.

First irradiation tests were performed by exposing test prototypes to 10 keV photons. 
No significant degradation in the analogue front-end performance were observed after a total
dose of 600 Mrad.


\vspace*{0.5cm}

\section{Conclusions}
A pixel front-end chain in 65~nm CMOS has been designed and tested. The key feature of the system 
is the usage of a clocked comparator with embedded offset calibration. 
When a signal is detected, the discriminator is automatically turned into a high-speed oscillator 
that allows fast, multi-bit charge encoding with time-over-threshold measurement. 




\begin{center}
\begin{figure}[!htpb]
\centering
\includegraphics[width=0.48\textwidth]{./pictures/plots/ENC_vs_Cin.pdf}
\caption{Measured ENC versus input capacitance for two different \break 
         values of feedback current.}  
\label{noise}
\end{figure}
\end{center}



\begin{center}
\begin{figure}[!htpb]
\centering
\includegraphics[width=0.48\textwidth]{./pictures/plots/untrimmed.pdf}
\caption{Measured discriminator threshold distribution before autozeroing.}  
\label{untrimmed}
\end{figure}
\end{center}





\begin{center}
\begin{figure}[!htpb]
\centering
\includegraphics[width=0.48\textwidth]{./pictures/plots/trimmed_v2.pdf}
\caption{Measured discriminator threshold distribution after autozeroing.}  
\label{trimmed}
\end{figure}
\end{center}









The noise is primarily dictated by the current chosen to discharged the CSA feedback capacitor, while no significant 
contribution induced by the digital signals present in the pixel cell has been observed. The design satisfies in term 
of rate capability, sensitivity and power consumption the stringent requirements envisaged for the Phase-II upgrades 
of the main LHC detectors.




\vspace*{1cm}



%% references
\begin{thebibliography}{10}
\small  % Use 9 point text.


\bibitem{Rossi2012} L. Rossi and O. Bruning,
                    \emph{High Luminosity Large Hadron Collider A Description for the European Strategy Preparatory Group},
					     CERN/ATS-2012-236, 2012

\bibitem{Bonacini2011} S. Bonacini et al.,
                       \emph{Characterization of a commercial 65~nm CMOS technology for SLHC applications},
                       JINST, 2011					

\bibitem{Christiansen2013} J. Christiansen and M. Garcia-Sciveres,
                           \emph{RD Collaboration Proposal: Development of Pixel Readout Integrated Circuits for 
						   Extreme Rate and Radiation},
                           CERN/LHCC-2013-008, 2013						   


\bibitem{Krummenacher1991} F. Krummenacher, 
                           \emph{Pixel detectors with Local Intelligence},
                           NIM~A, 1991 

\bibitem{Liu2010} C. C. Liu et al. ,
                  \emph{A 10-bit 50 MS/s SAR ADC with a Monotonic Capacitor Switching Procedure},
				  IEEE JSSC, 2010
						   


\bibitem{Razavi1992} B. Razavi and B. A. Wooley,
                     \emph{Design Techniques for High-Speed High-Resolution Comparators},
                     IEEE JSSC, 1992	   
\end{thebibliography}
\end{document}